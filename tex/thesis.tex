%%% Hlavní soubor. Zde se definují základní parametry a odkazuje se na ostatní části. %%%

%% Verze pro jednostranný tisk:
% Okraje: levý 40mm, pravý 25mm, horní a dolní 25mm
% (ale pozor, LaTeX si sám přidává 1in)
\documentclass[12pt,a4paper]{report}
\setlength\textwidth{145mm}
\setlength\textheight{247mm}
\setlength\oddsidemargin{15mm}
\setlength\evensidemargin{15mm}
\setlength\topmargin{0mm}
\setlength\headsep{0mm}
\setlength\headheight{0mm}
% \openright zařídí, aby následující text začínal na pravé straně knihy
\let\openright=\clearpage

%% Pokud tiskneme oboustranně:
% \documentclass[12pt,a4paper,twoside,openright]{report}
% \setlength\textwidth{145mm}
% \setlength\textheight{247mm}
% \setlength\oddsidemargin{15mm}
% \setlength\evensidemargin{0mm}
% \setlength\topmargin{0mm}
% \setlength\headsep{0mm}
% \setlength\headheight{0mm}
% \let\openright=\cleardoublepage

%% Použité kódování znaků: obvykle latin2, cp1250 nebo utf8:
\usepackage[utf8]{inputenc}

%% Ostatní balíčky
\usepackage{graphicx}
\usepackage{amsthm}
\usepackage{amsmath}
\usepackage{amssymb}
\usepackage{todonotes}

\theoremstyle{plain}
\newtheorem{theorem}{Theorem}[section]
\newtheorem{lemma}[theorem]{Lemma}
\newtheorem{prop}[theorem]{Proposition}
\newtheorem*{cor}{Corollary}

\theoremstyle{definition}
\newtheorem{definition}{Definition}[section]
\newtheorem{conj}{Conjecture}[section]
\newtheorem{exmp}{Example}[section]

\newtheorem{observation}{Observation}

%% Balíček hyperref, kterým jdou vyrábět klikací odkazy v PDF,
%% ale hlavně ho používáme k uložení metadat do PDF (včetně obsahu).
%% POZOR, nezapomeňte vyplnit jméno práce a autora.
\usepackage[ps2pdf,unicode]{hyperref}   % Musí být za všemi ostatními balíčky
\hypersetup{colorlinks=true}
\hypersetup{pdftitle=Název práce}
\hypersetup{pdfauthor=Tomáš Krejčí}

%%% Drobné úpravy stylu

% Tato makra přesvědčují mírně ošklivým trikem LaTeX, aby hlavičky kapitol
% sázel příčetněji a nevynechával nad nimi spoustu místa. Směle ignorujte.
\makeatletter
\def\@makechapterhead#1{
  {\parindent \z@ \raggedright \normalfont
   \Huge\bfseries \thechapter. #1
   \par\nobreak
   \vskip 20\p@
}}
\def\@makeschapterhead#1{
  {\parindent \z@ \raggedright \normalfont
   \Huge\bfseries #1
   \par\nobreak
   \vskip 20\p@
}}
\makeatother

% Toto makro definuje kapitolu, která není očíslovaná, ale je uvedena v obsahu.
\def\chapwithtoc#1{
\chapter*{#1}
\addcontentsline{toc}{chapter}{#1}
}

\begin{document}

% Trochu volnější nastavení dělení slov, než je default.
\lefthyphenmin=2
\righthyphenmin=2

%%% Titulní strana práce

\pagestyle{empty}
\begin{center}

\large

Charles University in Prague

\medskip

Faculty of Mathematics and Physics

\vfill

{\bf\Large MASTER THESIS}

\vfill

\centerline{\mbox{\includegraphics[width=60mm]{img/logo.pdf}}}

\vfill
\vspace{5mm}

{\LARGE Tomáš Krejčí}

\vspace{15mm}

% Název práce přesně podle zadání
{\LARGE\bfseries Title of the thesis}

\vfill

% Název katedry nebo ústavu, kde byla práce oficiálně zadána
% (dle Organizační struktury MFF UK)
Name of the department or institute

\vfill

\begin{tabular}{rl}

Supervisor of the master thesis: & First and last name \\
\noalign{\vspace{2mm}}
Study programme: & programme \\
\noalign{\vspace{2mm}}
Specialization: & specialization \\
\end{tabular}

\vfill

% Zde doplňte rok
Prague 2016

\end{center}

\newpage

%%% Následuje vevázaný list -- kopie podepsaného "Zadání diplomové práce".
%%% Toto zadání NENÍ součástí elektronické verze práce, nescanovat.

%%% Na tomto místě mohou být napsána případná poděkování (vedoucímu práce,
%%% konzultantovi, tomu, kdo zapůjčil software, literaturu apod.)

\openright

\noindent
Dedication.

\newpage

%%% Strana s čestným prohlášením k diplomové práci

\vglue 0pt plus 1fill

\noindent
I declare that I carried out this master thesis independently, and only with the cited
sources, literature and other professional sources.

\medskip\noindent
I understand that my work relates to the rights and obligations under the Act No.
121/2000 Coll., the Copyright Act, as amended, in particular the fact that the Charles
University in Prague has the right to conclude a license agreement on the use of this
work as a school work pursuant to Section 60 paragraph 1 of the Copyright Act.

\vspace{10mm}

\hbox{\hbox to 0.5\hsize{%
In ........ date ............
\hss}\hbox to 0.5\hsize{%
signature of the author
\hss}}

\vspace{20mm}
\newpage

%%% Povinná informační strana diplomové práce

\vbox to 0.5\vsize{
\setlength\parindent{0mm}
\setlength\parskip{5mm}

Název práce:
Název práce
% přesně dle zadání

Autor:
Tomáš Krejčí

Katedra:  % Případně Ústav:
Název katedry či ústavu, kde byla práce oficiálně zadána
% dle Organizační struktury MFF UK

Vedoucí diplomové práce:
Jméno a příjmení s tituly, pracoviště
% dle Organizační struktury MFF UK, případně plný název pracoviště mimo MFF UK

Abstrakt:
% abstrakt v rozsahu 80-200 slov; nejedná se však o opis zadání diplomové práce

Klíčová slova:
% 3 až 5 klíčových slov

\vss}\nobreak\vbox to 0.49\vsize{
\setlength\parindent{0mm}
\setlength\parskip{5mm}

Title:
% přesný překlad názvu práce v angličtině

Author:
Tomáš Krejčí

Department:
Název katedry či ústavu, kde byla práce oficiálně zadána
% dle Organizační struktury MFF UK v angličtině

Supervisor:
Jméno a příjmení s tituly, pracoviště
% dle Organizační struktury MFF UK, případně plný název pracoviště
% mimo MFF UK v angličtině

Abstract:
% abstrakt v rozsahu 80-200 slov v angličtině; nejedná se však o překlad
% zadání diplomové práce

Keywords:
% 3 až 5 klíčových slov v angličtině

\vss}

\newpage

%%% Strana s automaticky generovaným obsahem diplomové práce. U matematických
%%% prací je přípustné, aby seznam tabulek a zkratek, existují-li, byl umístěn
%%% na začátku práce, místo na jejím konci.

\openright
\pagestyle{plain}
\setcounter{page}{1}
\tableofcontents

\newpage
\openright
\pagestyle{plain}
\listoftodos

%%% Jednotlivé kapitoly práce jsou pro přehlednost uloženy v samostatných souborech
\chapter*{Introduction}
\addcontentsline{toc}{chapter}{Introduction}

1 to 3 pages

describe problem, structure at the end


\section{Cíl práce}

Na konci by měl být program, který je schopný dělat lokalizaci pomocí IMU a videa z kamery s kompenzací rolling shutter. Navíc by SW měl být schopen kompenzovat naakumulovanou chybu tím, že si bude tvořit interní mapu a v ní pomocí loop closure opravovat chyby. 

Další požadavky:
\begin{itemize}
\item Bude to balíček do ROSu (i když vývoj bude pravděpodobně probíhat jinak)
\item Otevřená licence
\item Musíme být schopni snadno ověřit výsledky (nějak snadno porovnatelné s ground truth)
\item Celá lokalizace by měla být energeticky nenáročná
\item Kalibrace by neměla vyžadovat žádné speciální znalosti (například časy uzávěrky apod.)
\end{itemize}

\section{Co očekávám, že bude moje přidaná hodnota}

Největší "akademická přidaná hodnota" by mělo být přidání globální mapy a loop closure k metodě založené na IMU. To pokud ví, nikdo nemá.

Další velká přidaná hodnota bude v otevřenosti samotného řešení. Opět, asi v současné době nejsou naimplementované žádné open-source řešení, která by fungovala takto sofistikovaně i s kamerou atp.

Integrace různých senzorů, stačí popsat jak se to udělá na nějaké hrubší úrovni.

\chapter{Preliminaries}


\chapter{Related works}

with structure!!!

more precise definitions

\verb+http://wiki.ros.org/robot_pose_ekf+ - Jednoduchá lokalizace pomocí EKF

\verb+http://wiki.ros.org/robot_localization+ - Možnosti pomocí UKF nebo EKF. Umí zakomponovat i GPS a VO, ale není nijak zvlášť sofistikované.

Li, M., Kim, B.H., Mourikis, A.: Real-time motion tracking on a cellphone using
inertial sensing and a rolling-shutter camera. In: IEEE International Conference
on Robotics and Automation (ICRA), pp. 4712–4719, May 2013

Tohle je ten hlavní článek o lokalizaci pomocí IMU a kamery s rolling shutter

\textbf{High-accuracy differential tracking of low-cost GPS receivers} 

\cite{hedgecock2013high}
\href{http://www.isis.vanderbilt.edu/projects/relativeGPS}{Web}\href{https://www.youtube.com/watch?v=BH149tSPrhs}{Youtube}

They implemented relative GPS position tracking which I can use as ground truth. They estimated accuracy of the system to be much better then I need. If I can make it work I can use it without any other justification (maybe?). Only think I need to make it work is two GPS receivers. It looks like they don't have to be same model or even manufacturer.

Method is implemented in Java. Only think I have to do is implement three interfaces. One for input GPS stream, one for byte array transfer over the network and one for data output. It seems to be reasonable easy to implement in matter of days (or maybe hours). 

\textbf{Real-time motion tracking on a cellphone using inertial sensing and a rolling-shutter camera}

\cite{li2013real} 
\href{http://www.ee.ucr.edu/~mli/RollingShutterVIO.html}{Web}

This is the main article. I have to study this one more deeply.

\textbf{Probabilistic robotics}

\cite{thrun2005probabilistic}

This is very good book recommanded by Barták and Obdržálek. They have nice derivation of Kalman Filter. I'll probably use this as primary source of teoretical part.

\textbf{Camera-Based Localization and Stabilization of a Flying Drone}

\cite{skoda2015camera}

This paper was recommanded to me by Barták. It promisses to do VO based camera localization using keypoints.

It also looks like there is nice derivation of EKF and it's usege for localization.

\textbf{Short-Term Motion Tracking Using Inexpensive Sensors}

\cite{matzner2015short}

There is really nice derivation for INS without rotation. It's very simple to understand. INS derivation uses some kind of rotation matrix which is very complicated. Maybe I can look at derivation of INS using quaternions for rotation. It might be simpler.

\textbf{Start Developing iOS Apps (Swift)}

\href{https://developer.apple.com/library/ios/referencelibrary/GettingStarted/DevelopiOSAppsSwift/}{Web}

This is very nice tutorial about programming iOS apps with Swift. I'll use it as starting point.
\chapter{My approach}

Definition of my problem. 

I'll try to describe my problem step by step. Starting with simplest navigation up to final result.

Maybe there'll be one chapter/section with description of final solution.

\section{iPhone specification}

It appears that iPhone 6s has MPU-6500 IMU \cite{iphone-spec-chipworks}. It should have precision of 250 (units?). Best gyro I was able to find was 200 from Sparkfun \cite{sparkfun-buying-guide}.

iPhone provides me only with accelerometer data in units G. This is bad for my application. I found on Wiki gravitation field in Frankfurt to be $9.81412$ measured 25. 1. 2016.
\todo{Where this number comes from}

There is possibility to at least estimate size of G-force by formula from wiki - Gravity of Earth.

Another way is to query Wolfram Alpha "gravitation acceleration, Prague". The result is $9.81373$. Based on EGM2008 12th order model, 223 meters above sea level

\section{Architecture}

I'll use my iPhone for computing all navigation tasks. It's good for
\begin{itemize}
\item I already have it
\item I don't have to resolve issues connected with hardware
\item IMU is correctly mounted on rigid body with camera (I've read a paper that says this was major source of error for them. Sadly I can't remember what the paper was. I only remember they shoot rocket and measured properties of IMU).
\item It is standard device. So everybody can easily imagine what is required to run this software
\item I have it everywhere with me
\end{itemize}

But there are also problems with this decision
\begin{itemize}
\item I can't controll precisely my HW
\item I'm limited to Apple's API
\item I don't know if apple do any preprocessing of data. According to \cite{apple-event-handling-guide-for-ios} it should be raw data.
\end{itemize}

For visualization I'll send all results over the network to ROS and then visualize them here. Main advantages are

\begin{itemize}
\item I don't have to study iOS development deeply.
\item Almost everything is ready for visualization in ROS.
\item ROS application might be quite simple
\item I can send GPS data too and then compute ground truth on ROS-device
\item I'm not limited to small display of iPhone
\item I can spend a lot of computing power on visualization
\item Network communication is relatively easy
\end{itemize}

Disadvantages
\begin{itemize}
\item I have to have another device for visualization
\end{itemize}



\section{Quaternions}

I'll probably use quaternions for INS. So it'd be great to have one place to keep review of quaternions at.

Quaternion $Q$ is
$$Q = \left( 1 q_1, i q_2, j q_3, k q_4 \right)$$

with following rules
\begin{align}
\label{eqn:quaternion-properties-1}
i^2 &= -1 & j^2 &= -1 & k^2 &= -1
\end{align}
\begin{equation}
\begin{align*}
\label{eqn:quaternion-properties-2}
i j &= k & j i &= -k \\
j k &= i & k j &= -i \\
k i &= j & i k &= -j
\end{align*}
\end{equation}


\textbf{Quaternion addition} is defined as
$$P+Q = (p_1 + q_1) + (p_2 + q_2)i + (p_3 + q_3)j + (p_4+q_4)k$$

\begin{observation}
Quaternoin addition is associative
$$P+Q=Q+P$$ 
\end{observation}

\begin{observation}
Quaternoin addition is commutative
$$(P+Q)+R=P+(Q+R)$$ 
\end{observation}

\begin{definition}[Quaternion multiplication]
\begin{align*}
PQ &= (p_1 + i p_2 + j p_3 + k p_4) (q_1 + i q_2 + j q_3 + k q_4) \\
&= p_1 q_1 - p_2 q_2 - p_3 q_3 - p_4 q_4 \\
& \quad + i (p_1 q_2 + p_2 q_1 + p_3 q_4 - p_4 q_3) \\
& \quad + j (p_1 q_3 + p_p q_1 + p_4 q_2 - p_2 q_4) \\
& \quad + k (p_1 q_4 + p_4 q_1 + p_2 q_3 - p_3 q_2)
\end{align*}
\end{definition}

Definition of quaternion multiplication is similar to polynomial multiplication wrt. properties \ref{eqn:quaternion-properties-1} and \ref{eqn:quaternion-properties-2}.

Maybe better memorizable if following table

\begin{center}
\begin{tabular}{c||c|c|c|c}
      & $q_1$ & $q_2$ & $q_3$ & $q_4$ \\
\hline \hline
$p_1$ & $1$   & $i$   & $j$   & $k$ \\
\hline
$p_2$ & $i$   & $-1$  & $k$   & $-j$ \\
\hline
$p_3$ & $j$   & $-k$  & $-1$  & $i$ \\
\hline
$p_4$ & $k$   & $j$   & $-i$  & $-1$
\end{tabular}
\end{center}

Which says that coefficient of term of $PQ$ with $p_i q_j$ can be found in table at row $p_i$ and column $q_j$.

\begin{definition}[Quaternion conjugates]
$$\overline{Q} = \overline{(q_1 + i q_2 + j q_3 + k q_4)} \equiv (q_i - i q_2 - j q_3 - k q_4)$$
\end{definition}

\begin{observation}
$(Q \overline{Q}) \in \mathbb{R}$
\end{observation}

\begin{observation}
$(Q + \overline{Q}) \in \mathbb{R}$
\end{observation}

\begin{observation}
$\overline{P+Q} = \overline{P} + \overline{Q}$
\end{observation}

\begin{observation}
$\overline{PQ} = \overline{Q} \cdot \overline{P}$
\end{observation}

\begin{definition}[Quaternion norm]
$$|Q| = \sqrt{q_1^2 + q_2^2 + q_3^2 + q_4^2}$$
\end{definition}

\begin{observation}
$Q \overline{Q} = \overline{Q} Q = |Q|^2$
\end{observation}

\subsection{Quaternions and rotation in $\mathbb{R}^3$}

\begin{theorem}{Rotation matrix theorem}
Rotation matrices are precisely those orthogonal matrices $A$ with $\det A = +1$.
\end{theorem}

Proof can be found in \href{https://www2.bc.edu/~reederma/Linalg17.pdf}{Mark Reeder's course Math 210 Linear Algebra, lesson 17}.

\todo{How to cite this correctly}

\begin{theorem}{Euler's theorem}
If $R$ is $3 \times 3$ matrix satistying $R^T R = R R^T = I$ and $\det R = +1$, then there is a non-zero vector $v$ satisfying $Rv = v$.
\end{theorem}

Proof of this theorem can be found in \cite{palais2007euler}.

\todo{Why rotation matrix in 3-space can be described by orthogonal matrix with determinant +1.}

This theorem tells us that we can describe every rotation in $\mathbb{R}^3$ as vector $v$ and angle $\theta$.

As a consequence of Euler's theorem we can say that any composition of rotations in $\mathbb{R}^3$ can be described in this way. This fact isn't immediately obvious. 

\include{03_experimental_results}
\chapter{Notes}

\section{Postup}

Tady nastíním postup takový, aby byl co možná nejkratší, inkrementální a jednotlivé kroky šly snadno ověřit a měly nějaký význam sami o sobě.

\begin{enumerate}

\item Jeste v prvni fazi posbirat co nejvice clanku, nakouknout a precist. Vytvorit si nejaky teoreticky zaklad.

\item Celý vývoj budu dělat nejprve na svém iPhonu, protože má vše potřebné, dostatečnou baterku a mám ho vždy u sebe. Mělo by mi to tedy pomoci usnadnit vývoj. Nese to s sebou drobnou komplikaci toho, že řešení musí být potom snadn portovatelné do ROSu. Navíc musím psát vše v C++ a pouze správně abstrahovat části, ktere interagují s okolním světem tak, abych je mohl snadno portovat pro ROS nebo jinou platformu.
[21 days]

\item Nejdříve bych měl být schopný interpretovat data z IMU a správně odečítat z kamery. To je komplikované zejména proto, že na iPhone se vyvíjí ve swiftu.

\item Další krok bude postavit kompletní INS pouze pomocí gyroskopu a akcelerometru. Tento krok bude jaksi "bokem", protože výsledné řešení bude používat EKF a bude fungovat trochu jinak. Myslím ale, že bude snadnější, když takto získám představu o tom, jak INS vlastně funguje. Potom mi to usnadní další vývoj.

\item (Volitelně: Nejdřív naimplementovat metodu z článku na který se odvolávají. Možná to bude snazší.) 

\item Potom bude cíl bude reimplementovat samotné EKF z článku doporučeného Filipem. Asi největší problém se zdá být samotné pochopení metody, protože se jedná o ne uplně obvyklou variantu EKF. 

\item Déle je krok validace této metody jako takové a srovnání s výsledky uvedenými v originálním článku. Pro tohle bude největší problém mít rozmyšlenou správnou metodu měření chyby. V principu mě napadají dvě varianty. Jedna pomocí dvou GPS (High-Accuracy Differential Tracking of Low-Cost GPS Receivers, W Hedgecock, M Maroti, J Sallai, P Volgyesi, A Ledeczi) a potom pomocí loop closure, kde se ale dozvím jen celkovou naakumulovanou chybu.

\item Potom vzít výsledky z předchozího kroku a přidat k nim tvorhu globální mapy tak, jak ji dělá ORB SLAM (ORB-SLAM: A Versatile and Accurate Monocular SLAM System). 

\item Nakonec porovnání s předchozí variantou o kolik se mi podařilo zpřesnit navigaci.
\end{enumerate}

\section{Harmonogram}

\begin{itemize}
\item \textbf{začátek května} Mít definované přesné zadání práce.
\item \textbf{do 28. 7. 2016} Odevzdání bakalářských a diplomových prací pro podzimní termín státních závěrečných zkoušek 
Přihlášení se k podzimnímu termínu bakalářských a magisterských státních závěrečných zkoušek 
Uzavření studia závěrečných ročníků bakalářského a magisterského studia – kontrola splnění všech podmínek pro připuštění k podzimnímu termínu SZZ 
\end{itemize}



\section{29. 2. 2016}
\begin{itemize}
\item \textit{Ask Filip Matzner about his Master's thesi}

He'll do neuroevolution.

\item \textit{Ask Filip for his sources}

Filip sent me his sources. I'll have a look at it. Thanks Filip! :)

\item \textit{Look at web of paper publisher if there are any additional files}

It looks like there are no additional files at \href{http://ieeexplore.ieee.org/xpl/abstractSimilar.jsp?arnumber=6631248&tag=1}{web}.


\item \textit{Contact Tomas Prochazka (tms.prochazka@gmail.com) about his visualization in thesis}

I contacted him. I meet him and talked about visualization and his work in general. He uses Java only. For visualization he use JFreeChart library. It looks like quite powerful library for static plots. However for animation and 3D it doesn't seems to be best choice. I'll probably not use any of this.

\item \textit{Bakalářka Jan Škoda, bude mít odkaz na článek ze kterého se vycházelo, možná má nějaký článek. Stabilizace drona na základě VO feature based.}

I've downloaded the paper. I've added section below with comments about the paper.

\end{itemize}

\section{7. 3. 2016}
\begin{itemize}
\item \textit{How to deal with HW. Is there any kind of support? Does anybody have two same GPS receivers?}

Ask Obdrzalek about HW. He may borrow me somethink.

\item \textit{Can I ask library to buy book \cite{titterton2004strapdown}?}

Bartak will think about this. Maybe he will buy it from grant or ask library.

\item \textit{How to properly cite web pages.}

I don't have to strictly follor ISO 960. I'll use the best option for citations I can find.

\end{itemize}










% Ukázka použití některých konstrukcí LateXu (odkomentujte, chcete-li)
% %%% Ukázka použití některých konstrukcí LaTeXu

\subsection{Ukázka \LaTeX{}u}
\label{ssec:ukazka}

This short subsection serves as an~example of basic \LaTeX{} constructs,
which can be useful for writing a~thesis.

Let us start with lists:

\begin{itemize}
\item The logo of Matfyz is displayed in figure~\ref{fig:mff}.
\item This is subsection~\ref{ssec:ukazka}.
\item Citing literature~\cite{lamport94}.
\end{itemize}

Different kinds of dashes:
red-black (short),
pages 16--22 (middle),
$45-44$ (minus),
and this is --- as you could have expected --- a~sentence-level dash,
which is the longest.
(Note that we have follwed \verb|a| by a~tilde instead of a~space
to avoid line breaks at that place.)

\newtheorem{theorem}{Theorem}
\newtheorem*{define}{Definition}	% Definice nečíslujeme, proto "*"

\begin{define}
A~{\sl Tree} is a connected graph with no cycles.
\end{define}

\begin{theorem}
This theorem is false.
\end{theorem}

\begin{proof}
False theorems do not have proofs.
\end{proof}

\begin{figure}
	\centering
	\includegraphics[width=30mm]{../img/logo.eps}
	\caption{Logo of MFF UK}
	\label{fig:mff}
\end{figure}


\include{90_conclusion}

%%% Seznam použité literatury
%%% Seznam použité literatury je zpracován podle platných standardů. Povinnou citační
%%% normou pro diplomovou práci je ISO 690. Jména časopisů lze uvádět zkráceně, ale jen
%%% v kodifikované podobě. Všechny použité zdroje a prameny musí být řádně citovány.

\bibliographystyle{unsrt}
\bibliography{bibliography}

%\def\bibname{Bibliography}
%\begin{thebibliography}{99}
%\addcontentsline{toc}{chapter}{\bibname}

%\bibitem{lamport94}
%  {\sc Lamport,} Leslie.
%  \emph{\LaTeX: A Document Preparation System}.
%  2. vydání.
%  Massachusetts: Addison Wesley, 1994.
%  ISBN 0-201-52983-1.

%\end{thebibliography}


%%% Tabulky v diplomové práci, existují-li.
\chapwithtoc{List of Tables}

%%% Použité zkratky v diplomové práci, existují-li, včetně jejich vysvětlení.
\chapwithtoc{List of Abbreviations}

%%% Přílohy k diplomové práci, existují-li (různé dodatky jako výpisy programů,
%%% diagramy apod.). Každá příloha musí být alespoň jednou odkazována z vlastního
%%% textu práce. Přílohy se číslují.
\chapwithtoc{Attachments}

\openright
\end{document}
