\chapter{My approach}

Definition of my problem. 

I'll try to describe my problem step by step. Starting with simplest navigation up to final result.

Maybe there'll be one chapter/section with description of final solution.

\section{iPhone specification}

It appears that iPhone 6s has MPU-6500 IMU \cite{iphone-spec-chipworks}. It should have precision of 250 (units?). Best gyro I was able to find was 200 from Sparkfun \cite{sparkfun-buying-guide}.

iPhone provides me only with accelerometer data in units G. This is bad for my application. I found on Wiki gravitation field in Frankfurt to be $9.81412$ measured 25. 1. 2016.
\todo{Where this number comes from}

There is possibility to at least estimate size of G-force by formula from wiki - Gravity of Earth.

Another way is to query Wolfram Alpha "gravitation acceleration, Prague". The result is $9.81373$. Based on EGM2008 12th order model, 223 meters above sea level

\section{Architecture}

I'll use my iPhone for computing all navigation tasks. It's good for
\begin{itemize}
\item I already have it
\item I don't have to resolve issues connected with hardware
\item IMU is correctly mounted on rigid body with camera (I've read a paper that says this was major source of error for them. Sadly I can't remember what the paper was. I only remember they shoot rocket and measured properties of IMU).
\item It is standard device. So everybody can easily imagine what is required to run this software
\item I have it everywhere with me
\end{itemize}

But there are also problems with this decision
\begin{itemize}
\item I can't controll precisely my HW
\item I'm limited to Apple's API
\item I don't know if apple do any preprocessing of data. According to \cite{apple-event-handling-guide-for-ios} it should be raw data.
\end{itemize}

For visualization I'll send all results over the network to ROS and then visualize them here. Main advantages are

\begin{itemize}
\item I don't have to study iOS development deeply.
\item Almost everything is ready for visualization in ROS.
\item ROS application might be quite simple
\item I can send GPS data too and then compute ground truth on ROS-device
\item I'm not limited to small display of iPhone
\item I can spend a lot of computing power on visualization
\item Network communication is relatively easy
\end{itemize}

Disadvantages
\begin{itemize}
\item I have to have another device for visualization
\end{itemize}



\section{Quaternions}

I'll probably use quaternions for INS. So it'd be great to have one place to keep review of quaternions at.

Quaternion $Q$ is
$$Q = \left( 1 q_1, i q_2, j q_3, k q_4 \right)$$

with following rules
\begin{align}
\label{eqn:quaternion-properties-1}
i^2 &= -1 & j^2 &= -1 & k^2 &= -1
\end{align}
\begin{equation}
\begin{align*}
\label{eqn:quaternion-properties-2}
i j &= k & j i &= -k \\
j k &= i & k j &= -i \\
k i &= j & i k &= -j
\end{align*}
\end{equation}


\textbf{Quaternion addition} is defined as
$$P+Q = (p_1 + q_1) + (p_2 + q_2)i + (p_3 + q_3)j + (p_4+q_4)k$$

\begin{observation}
Quaternoin addition is associative
$$P+Q=Q+P$$ 
\end{observation}

\begin{observation}
Quaternoin addition is commutative
$$(P+Q)+R=P+(Q+R)$$ 
\end{observation}

\begin{definition}[Quaternion multiplication]
\begin{align*}
PQ &= (p_1 + i p_2 + j p_3 + k p_4) (q_1 + i q_2 + j q_3 + k q_4) \\
&= p_1 q_1 - p_2 q_2 - p_3 q_3 - p_4 q_4 \\
& \quad + i (p_1 q_2 + p_2 q_1 + p_3 q_4 - p_4 q_3) \\
& \quad + j (p_1 q_3 + p_p q_1 + p_4 q_2 - p_2 q_4) \\
& \quad + k (p_1 q_4 + p_4 q_1 + p_2 q_3 - p_3 q_2)
\end{align*}
\end{definition}

Definition of quaternion multiplication is similar to polynomial multiplication wrt. properties \ref{eqn:quaternion-properties-1} and \ref{eqn:quaternion-properties-2}.

Maybe better memorizable if following table

\begin{center}
\begin{tabular}{c||c|c|c|c}
      & $q_1$ & $q_2$ & $q_3$ & $q_4$ \\
\hline \hline
$p_1$ & $1$   & $i$   & $j$   & $k$ \\
\hline
$p_2$ & $i$   & $-1$  & $k$   & $-j$ \\
\hline
$p_3$ & $j$   & $-k$  & $-1$  & $i$ \\
\hline
$p_4$ & $k$   & $j$   & $-i$  & $-1$
\end{tabular}
\end{center}

Which says that coefficient of term of $PQ$ with $p_i q_j$ can be found in table at row $p_i$ and column $q_j$.

\begin{definition}[Quaternion conjugates]
$$\overline{Q} = \overline{(q_1 + i q_2 + j q_3 + k q_4)} \equiv (q_i - i q_2 - j q_3 - k q_4)$$
\end{definition}

\begin{observation}
$(Q \overline{Q}) \in \mathbb{R}$
\end{observation}

\begin{observation}
$(Q + \overline{Q}) \in \mathbb{R}$
\end{observation}

\begin{observation}
$\overline{P+Q} = \overline{P} + \overline{Q}$
\end{observation}

\begin{observation}
$\overline{PQ} = \overline{Q} \cdot \overline{P}$
\end{observation}

\begin{definition}[Quaternion norm]
$$|Q| = \sqrt{q_1^2 + q_2^2 + q_3^2 + q_4^2}$$
\end{definition}

\begin{observation}
$Q \overline{Q} = \overline{Q} Q = |Q|^2$
\end{observation}

\subsection{Quaternions and rotation in $\mathbb{R}^3$}

\begin{theorem}{Rotation matrix theorem}
Rotation matrices are precisely those orthogonal matrices $A$ with $\det A = +1$.
\end{theorem}

Proof can be found in \href{https://www2.bc.edu/~reederma/Linalg17.pdf}{Mark Reeder's course Math 210 Linear Algebra, lesson 17}.

\todo{How to cite this correctly}

\begin{theorem}{Euler's theorem}
If $R$ is $3 \times 3$ matrix satistying $R^T R = R R^T = I$ and $\det R = +1$, then there is a non-zero vector $v$ satisfying $Rv = v$.
\end{theorem}

Proof of this theorem can be found in \cite{palais2007euler}.

\todo{Why rotation matrix in 3-space can be described by orthogonal matrix with determinant +1.}

This theorem tells us that we can describe every rotation in $\mathbb{R}^3$ as vector $v$ and angle $\theta$.

As a consequence of Euler's theorem we can say that any composition of rotations in $\mathbb{R}^3$ can be described in this way. This fact isn't immediately obvious. 
